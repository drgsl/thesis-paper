\documentclass[12pt]{report}

\def\blankpage{%
      \clearpage%
      \thispagestyle{empty}%
      \addtocounter{page}{-1}%
      \null%
      \clearpage} 


% includes
\usepackage{geometry}           % page size
\usepackage[utf8]{inputenc}     % encoding
\usepackage{palatino}           % font
\usepackage[english]{babel}     % language
\usepackage{graphicx}           % images
\usepackage{indentfirst}        % indentation
\usepackage[nottoc]{tocbibind}  % table of contents style
\usepackage[unicode]{hyperref}  % references from the table of contents
\usepackage{wrapfig}            % for the small images that have text around
\usepackage[rightcaption]{sidecap}
\usepackage{amsmath}            % for math cases in functions

\usepackage{xcolor}

% includes options
\geometry{  a4paper,            % scientific thesis standard
            left=3cm,
            right=2cm,
            top=2cm,
            bottom=2cm,
 }
\graphicspath{{src/img/}}        % path where the images are located
\setlength{\parindent}{1cm}     % paragraph indentation

% other options
\linespread{1.5}                % space between lines
\renewcommand*\contentsname{Table of contents}    % table of contents name




% FOR PROGRAMMING SNIPPETS


\usepackage{listings} % For code listings
\usepackage{xcolor}   % For coloring code

% Define colors for code highlighting
\definecolor{codegreen}{rgb}{0,0.6,0}
\definecolor{codegray}{rgb}{0.5,0.5,0.5}
\definecolor{codepurple}{rgb}{0.58,0,0.82}
\definecolor{backcolour}{rgb}{0.95,0.95,0.92}

% Define C++ code style
\lstset{
    language=C++,
    backgroundcolor=\color{backcolour},   
    commentstyle=\color{codegreen},
    keywordstyle=\color{blue},
    numberstyle=\tiny\color{codegray},
    stringstyle=\color{codepurple},
    basicstyle=\footnotesize\ttfamily,
    breakatwhitespace=false,
    breaklines=true,
    captionpos=b,
    keepspaces=true,
    numbers=left,
    numbersep=5pt,
    showspaces=false,
    showstringspaces=false,
    showtabs=false,
    tabsize=2,
    frame=single
}





% the document content
\begin{document}
    % macros (global)
    \newcommand{\university}    {"Alexandru-Ioan Cuza" University}
\newcommand{\universityg}   {Universității "Alexandru-Ioan Cuza" din Iași} % genitive
\newcommand{\faculty}       {Computer Science}
\newcommand{\facultyg}      {Facultății de informatică} % genitive
\newcommand{\speciality}    {Computer Science}
\newcommand{\promotion}     {2024}                                  %<---------

\newcommand{\thesistype}    {Thesis Paper}
\newcommand{\thesistitle}   {Machine Learning Compatible Game Engine}    %<---------

\newcommand{\authorlast}    {Bobu}                               %<---------
\newcommand{\authorfirst}   {Dragoş-Andrei}
\newcommand{\authornamefl}  {\authorfirst \space \authorlast} % first name first
\newcommand{\authornamelf}  {\authorlast \space \authorfirst} % last name first
\newcommand{\authorbirth}   {06 iulie 2001}                      %<---------
\newcommand{\authoraddress} {România, jud. Iași, mun. Iași, Valea lupului, str. Nufarului, nr 18} %<---------
\newcommand{\authorcnp}     {5010607374511}                         %<---------

\newcommand{\session}       {July, 2024}                       %<---------
\newcommand{\coordinator}   {Dr. Cosmin Varlan, Lecturer}               %<---------

\newcommand{\dottedline}    {............................}

    
    % front-matter
    \pagenumbering{gobble}








    \blankpage
    % define the cover page
\begin{titlepage}
    \begin{center}
        % the university and faculty
        \large
        \MakeUppercase{\university}
        
        \LARGE
        \textbf{\MakeUppercase{\faculty}}
        
        % the faculty logo
        \vspace{1cm}
        \includegraphics[width=0.3\textwidth]{logoFii.png}
        
        % thesis title
        \vspace{1cm}
        \Large
        \MakeUppercase{\thesistype}
        
        \vspace{0.5cm}
        \LARGE
        \textbf{\thesistitle}
        
        % author
        \vspace{2cm}
        \Large
        author
        
        \vspace{0.5cm}
        \LARGE
        \textbf{\authornamefl}
        
        % session
        \vfill
        \Large
        \textbf{session:} \session
        
        % scientific coordinator
        \vspace{2cm}
        \Large
        scientific coordinator
        
        \vspace{0.5cm}
        \LARGE
        \textbf{\coordinator}
    \end{center}
\end{titlepage}
    \blankpage
    \input{src/front/title}
    \blankpage

    \setcounter{page}{1}
    \pagenumbering{arabic}
    






    \blankpage
    

Computer Software is just a bunch of tools orchestrated in just the right manner.


Similarly to this, a game engine is just a software framework of computer graphics components.


The Game Editor is what allows users to custom build new tools that work seemingly with the already existing SOLID engines.



 




    % table of contents
    \tableofcontents






    \part{Introduction}
      \chapter*{Introduction}
        \section*{Motivation}
          







This project wished to evolve into a solid service with many endpoints that users can benefit from it's unique collection of components in order to build graphical experiences.



This edition's update journal will iterate over the implementation status of the first building blocks in the toolkit. 














        \section*{Introduction}
          









\fancyhead{} % clear all footer fields

\fancyhead[L]{INTRODUCTION}
\fancyhead[C]{}
\fancyhead[R]{MOTIVATION}
Non-Player Characters (NPCs) serve as guides for players and act as extensions of the game designers within the game world. Consequently, it is crucial for NPCs to have fluid dialogue that maintains the illusion of choice without easily breaking it.

Currently, dialogue trees are the predominant solution for managing NPC interactions. While effective, dialogue trees can still feel somewhat rigid and may disrupt the immersion when players are limited to selecting from a set of predefined responses.





















          










This implementation required extensive study of multiple computer science fields. 
Out of which i will name a few: 

\begin{itemize}
  \item Computer Graphics
  \item Game     Development
  \item Machine  Learning
\end{itemize}


% \includegraphics[width=0.8\textwidth]{courses.png}



Beside those, courses such as the following had made me an agile handyman that has his toolbox in order:
\begin{itemize}
  \item oop, data structures and Software Engineering
  \item python programming, plp and Logics
\end{itemize}

% \includegraphics[width=0.8\textwidth]{courses2.png}



















        \section*{Game Engines}
          In this section, we will inspect the criteria to which a piece of software needs to qualify in order to be considered part of the game engines set.
          




subsection*{Natural Language Understanding}

\paragraph*{Wikipedia}
"A game engine is a software framework primarily designed for the development of video games and generally includes relevant engines."



\subsection*{Formal Definition}
  Let \( S \) be a piece of software, input for a function \( isGameEngine \) that outputs a binary response. 
 \[
    \text{isGameEngine} (\text{Software } S) =
    \text{isFramework}(S) \land (\text{S::Components} \subseteq \text{GE::E})
\] 

  where GameEngine::Engines
  represents the collection of all possible engines in all possible game engines, as such:
  \[
  \text{GE::E} = 
  \[
    \{
      ...
    \text{Render}, 
    \text{Script}, 
    \text{Physics}, 
    \text{AI}, 
    \text{SFX}, 
    \text{Robotics}
    ...
   \}
  \end{cases}
  \]






















    \part{Implementation}
      \chapter*{Implementation}
        


\subsection*{Results}

\begin{lstlisting}[caption={DVD\_Logo\_Bouncer Example Code}]
#include "../../engine/engine.h"  
int main()
{
  RenderEngine::setStart(start); RenderEngine::setUpdate(update); RenderEngine::Enabled(true);
}
GameObject dvd; GameObject vWalls[2]; GameObject hWalls[2];
void start()
{
  dvd.rigidbody.velocity = Vector3(0.01, 0.001);

  vWalls[0].transform.position = Vector3(-1, 0); vWalls[0].transform.scale = Vector3(0.01, 2);
  ...
  hWalls[1].transform.position = Vector3(0, 1); hWalls[1].transform.scale = Vector3(2, 0.01);
}
void update()
{
  RenderEngine::background(0);
  dvd.transform.position += dvd.rigidbody.velocity;
  dvd.show();
  for (GameObject& wall : vWalls)
    if(dvd.collides(wall)) 
      dvd.rigidbody.velocity.x = -dvd.rigidbody.velocity.x;

  for (GameObject& wall : hWalls)
    if(dvd.collides(wall)) 
      dvd.rigidbody.velocity.y = -dvd.rigidbody.velocity.y;

  if(GameEngine::Input::getKeyDown(GameEngine::KEY_R))
    dvd.transform.position = Vector3::zero;
}
\end{lstlisting}

\subsection*{Analysis}

\begin{minipage}[t]{0.5\linewidth}
    \textbf{start Function Description:}
    \begin{itemize}
        \item \textbf{DVD Velocity:} $(0.01, 0.001)$
        \item \textbf{Vertical Walls:}
        \begin{itemize}
            \item Left: $(-1, 0)$
            \item Right: $(1, 0)$
        \end{itemize}
        \item \textbf{Horizontal Walls:}
        \begin{itemize}
            \item Bottom: $(0, -1)$
            \item Top: $(0, 1)$
        \end{itemize}
    \end{itemize}
\end{minipage}%
\begin{minipage}[t]{0.5\linewidth}
    \textbf{update Function Description:}
    \begin{itemize}
        \item \textbf{Collision Detection:}
        \begin{itemize}
            \item Let $A$ and $B$ be two axis-aligned squares with:
            \begin{itemize}
                \item Center of $A$: $(x_A, y_A)$, Side length: $s_A$
                \item Center of $B$: $(x_B, y_B)$, Side length: $s_B$
            \end{itemize}
            \item Collision occurs if:
            \[
            (x_A - x_B)^2 + (y_A - y_B)^2 \leq \left(\frac{s_A + s_B}{2}\right)^2
            \]
        \end{itemize}
    \end{itemize}
\end{minipage}




        \section*{Components}
          In this section, we will inspect each component present in the implementation.
          
\chapter*{Mathematical Framework}

This mathematical framework is very important because it assures that the phigs requirements can be fulfiled and how to.

\section*{Homogenous Representation}

Intuitively, for storing a 3D point one might think about using a vector of length 3 ( maybe call them x, y, z ) and have a good day.
Well, this datastructure is good \emph{enough} for most cases.
There is, of course, one little edge case in one of the things we must be able to do with this DataStructure that would benefit of storing a 3D point as a 4D Vector (x, y, z, \textbf{\emph{w}}).

\textbf{Please do not be afraid of the \emph{w}.  Since \emph{w} is defined by a pretty straight-forward formula.} 

w=1, if point and 
w=0, if arrow

% TODO: MATH FORMULA FOR w = 1, if point w = 0, if arrow

\subsection*{Vector Multiplication}
\subsection*{Matrix Multiplication}

\section*{Coordinate Systems}

There are two main coordinate systems:
\begin{itemize}
  \item Cartesian Coordinates
  \item Polar     Coordinates 
\end{itemize}

Formulas for converting from one to another:
\begin{itemize}
  \item from Polar     to Cartesian
  \begin{itemize}
    \item cos(x)
    \item sin(x)
  \end{itemize}
  \item from Cartesian to Polar
  \begin{itemize}
    \item x
    \item y
  \end{itemize}
\end{itemize}


In the following, we will take a look over the possible operation that are extensively used:

In most of the cases, the equasions are really easy.

\subsection*{Translation}
\subsection*{Scalation}
\subsection*{Rotation}
\subsubsection*{Euler-Lock}
And it would've been all so easy if it weren't for you! 

Euler Lock is a problem that ocurs when we try to use euler coordinates in rotation aplications. This problem can become extremily dangerous when solving robotics solutions where you can't afford to ???

\subsubsection*{Quaternions}
For eliminating the euler-lock problem, quaternions are used. Quaternions are ??

Formulas:








          


\subsection*{Data Structure}







\begin{lstlisting} [caption={Color Declaration}]
class Color32
{
  public:
    unsigned int r, g, b, a;

    Color32(double grayscale) 
           : Color(grayscale, grayscale, grayscale, 1.0f) { }
    Color32(double _r, double _g, double _b, double _a = 1.0f) {  }

\end{lstlisting}


\begin{lstlisting}  [caption={Input Handler}]
  ...
  static std::map<unsigned char, bool> KeyUpTable;
  static std::map<unsigned char, bool> KeyTable;
  static std::map<unsigned char, bool> KeyDownTable;

  static void updateKeyState(unsigned char key, bool state = true);
  static void resetKeyStates();

class Input
{
  public:
    static bool getKeyDown(GameEngine::KeyCode key);
    static bool getKeyDown(GameEngine::KeyCode key);
    static bool getKeyUp(GameEngine::KeyCode key);

    static bool getAxisRaw(std::string axisName);
};

enum Keycode 
{
  KEY_A = 'a',
  ...
}

\end{lstlisting}



\pagebreak


\subsection*{Rendering Pipeline}

\begin{lstlisting} [caption={Renderer Declaration}]
class opengl {
  private:
    static void DisplayFunc;  static void ReshapeFunc;
    static void KeyboardFunc; static void KeyboardUpFunc; static void MouseFunc;
  public:
    void setAwake (void (*func)()); void setStart(void (*func)());
    void setUpdate(void (*func)()); void setFixedUpdate(void (*func)());

    void background(Color c);           void fill();      void noFill(); 
    void stroke(Color c);               void strokeWeight(double weight);
    void point(Vector3 pos);            void line(Vector3 start, Vector3 end);
    void rect(Vector3 bL, Vector3 tR);  void circle(Vector3 c, double r, double seg=1000); 
};
\end{lstlisting}

% // Point function
% point: R^2 -> Render, 
%                             point(x):
%                               glBegin(GL_POINTS)
%                               glVertex2f(x)
%                               glEnd()

\begin{lstlisting}[caption={Renderer Definition}]
...
line: (R^2, R^2) -> Render,
                              glBegin(GL_LINES)
                                glVertex2f(x1, y1)
                                glVertex2f(x2, y2)
                              glEnd()
rect: (R^2, R^2) -> Render
                              glBegin(GL_QUADS)
                                glVertex2f(x1, y1)
                                glVertex2f(x2, y1)
                                glVertex2f(x2, y2)
                                glVertex2f(x1, y2)
                              glEnd()
circle: (R^2, R) -> Render
                              glBegin(GL_TRIANGLE_FAN)
                                glVertex2f(xc + r * cos(i), yc + r * sin(i))
                                for i in [0, 2pi, 2pi/segments]:
                                  glVertex2f(xc + r * cos(i), yc + r * sin(i))
                              glEnd()
\end{lstlisting}
% background: Color -> Render
%                                 glClearColor(r, g, b, a)
%                                 glClear(GL_COLOR_BUFFER_BIT)
% stroke: Color -> State
%                                 glColor4f(r, g, b, a)



\pagebreak

          \input{src/content/app/components/probability}
          

  
  \[
    \text{Directory } D := \{d_1, d_2, \ldots, d_n\}, \quad \text{where } \text{TypeOf}(d_i) \in \{\text{File}, \text{Directory}\}
  \]
  
  \[
    \forall \ module_i \in \text{frontend} \Rightarrow 
    module_i \supseteq \{\text{init.py}, \text{main.py}, \text{cli.py}\}
  \]
  
  Specific to this implementation, we have the following modules and sub-modules:
  
  \[
    (\exists \ \text{Editor} \in \text{Modules}). \ \text{Editor.submodules} \
    \rightarrow_{\text{depend on}} \ \text{PyQt}
  \]
  
  \[
    (\exists \ \text{Editor} \in \text{Modules}). \ \text{Editor.submodules} \
    \supseteq \{\text{Hierarchy}, \text{Scene View}, \text{Inspector}, \text{Assets}, \text{Terminal}\}
  \]
  
  \[
    (\exists \ \text{Assets} \in \text{Editor.submodules}). \ \text{Assets} \rightarrow_{\text{modifies}} \{\text{Project Tree}\} 
  \]
  
  \[
    (\exists \ \text{Hierarchy} \in \text{Editor.submodules}). \ \text{Hierarchy} \rightarrow_{\text{modifies}} \ \text{JSON\_Scene\_File}
  \]
  
  \[
    (\exists \ \text{Inspector} \in \text{Editor.submodules}). \ \text{Inspector} \rightarrow_{\text{modifies}} \ \text{JSON\_Scene\_File}
  \]
  
  The interactions between different modules can be represented using functions and mappings. Let \( f: E \to H \) represent the function mapping elements from the Editor to the Hub. 
  Similarly, \( g: H \to S_p \) represents the mapping from Hub to Splash.
  
  \[
  \begin{aligned}
    \text{user} &\xrightarrow{\text{launch}} \text{splash}     
         \xrightarrow{\text{launch}} \text{Hub}          
         \xrightarrow{\text{launch}} \text{Editor} \\
  \end{aligned}
  \]
 
  \[
  \begin{aligned}
        \text{Editor}  &\xrightarrow{\text{launch}} \{\text{Hierarchy}, \text{SceneView}, \text{Inspector}, \text{Assets}\}
  \end{aligned}
  \]

  \[
  \text{SceneView} \xrightarrow{\text{launch}} (\text{c++ } \land \text{python}) \ \text{Runner} \xrightarrow{\text{launch}} \{\text{opengl}, \text{scipy}\}
  \]
  
  \[
  \text{Hierarchy} \xrightarrow{\text{reads}} \{\text{active\_scene.json}\}
  \]
  
  \[
  \text{Inspector} \xrightarrow{\text{reads}} \{\text{active\_game\_object.json}\}
  \]
  
  \[
  \text{Assets} \xrightarrow{\text{reads}} \{\text{FileTreeOf(projectPath)}\}
  \]
  
  \[
  (\text{user} \land \text{Hub}) \lor (\text{user} \land \text{Editor}) \lor (\text{user} \land \text{CLInterface}) \xrightarrow{\text{Request}} (\text{FileManager} \lor \text{SceneManager})
  \xrightarrow{\text{response}}
  \]
  







  \pagebreak

\section*{Submodule Communication}

In the game engine, submodules within the frontend communicate using well-defined processes to ensure a coherent and synchronized user experience. Each submodule is designed to handle specific tasks, and they interact with each other through a series of function calls, events, and data sharing mechanisms.

\subsection*{Communication Mechanisms}

\begin{itemize}
    \item \textbf{Function Calls:} Direct function calls are used when a submodule needs to invoke a specific operation in another submodule. For example, the Scene View might call functions in the Inspector to update the properties of a selected object.
    \item \textbf{Event System:} An event-driven architecture allows submodules to subscribe to and broadcast events. When an event occurs (e.g., a user selects an object in the Hierarchy), an event is broadcasted to all interested submodules (e.g., Scene View and Inspector).
    \item \textbf{Shared Data Structures:} Submodules often share data structures, such as the JSON scene file. When one submodule modifies this data (e.g., the Hierarchy reorders objects), the changes are reflected across all submodules that read from the same data.
\end{itemize}

\subsection*{Example Workflow}

Consider a scenario where the user selects a game object in the Hierarchy, and this selection needs to be reflected in the Scene View and Inspector:

\begin{enumerate}
    \item \textbf{Hierarchy:} The user clicks on an object in the Hierarchy submodule.
    \item \textbf{Event Broadcast:} The Hierarchy submodule broadcasts a "selection changed" event.
    \item \textbf{Scene View:} The Scene View submodule receives the event and highlights the selected object.
    \item \textbf{Inspector:} The Inspector submodule receives the event and displays the properties of the selected object for editing.
\end{enumerate}

\subsection*{Inter-Process Communication (IPC)}

For more complex interactions, especially those that might involve asynchronous operations or different processes, Inter-Process Communication (IPC) mechanisms are used:

\begin{itemize}
    \item \textbf{Message Passing:} Submodules send messages to each other to request actions or share information. This can be implemented using various IPC methods such as sockets, message queues, or shared memory.
    \item \textbf{Remote Procedure Calls (RPC):} One submodule can invoke functions in another submodule as if they were local, even though they might be running in separate processes. This abstraction simplifies communication and coordination.
\end{itemize}

% \subsection*{Synchronization}

% To maintain data consistency and avoid race conditions, synchronization mechanisms such as locks, semaphores, and mutexes are employed. These ensure that only one submodule can modify shared data at a time, preserving the integrity of the data.

% \subsection*{Summary}

% The communication between frontend submodules in the game engine is a blend of direct function calls, event-driven updates, shared data structures, and IPC mechanisms. This robust communication framework allows for a highly interactive and responsive user interface, where changes in one submodule are promptly reflected across all relevant parts of the application.













  % \[
  % \xrightarrow{\text{request}} (\text{FileManager} \lor \text{SceneManager})
  % \xrightarrow{\text{response}}
  % \]
  % 

% \begin{quote}
% The frontend component is organized 
% into several directories and files. 

% It includes functionalities for 
% the editor, hub, and splash interfaces. 
% \end{quote}


% \begin{math}
%   Directory \ D \ :=  \{d_1, d_2, \ldots, d_n\}
%   , 
%   where \ \ \TypeOf (\( d_i \)) \in \{File, Directory\}
% \end{math}


% \begin{math}
%   \forall module_i \in frontend \Rightarrow 
%   module_i 
%   \supseteq
%   \{init.py, main.py, cli.py\}
% \end{math}

% Specific to this implementation, 
% we have the following modules and sub-modules:


% \begin{math}
%   (\exists \ Editor \in Modules).
%   Editor.submodules \
%   \rightarrow_{depend\ on} PyQt
% \end{math}

% \begin{math}
%   (\exists \ Editor \in Modules).
%   Editor.submodules \
%   \supseteq \{ Hierarchy, Scene \ View, Inspector, Assets, Terminal \}
% \end{math}


% \begin{math}
%   (\exists \ Assets \in Editor.submodules).
%   \ Assets \rightarrow_{modifies} \{ \ Project \ Tree \} 
% \end{math}

% \begin{math}
%   (\exists \ Hierarchy \in Editor.submodules).
%   \ Hierarchy \rightarrow_{modifies} \ JSON\_Scene\_File
% \end{math}

% \begin{math}
%   (\exists \ Inspector \in Editor.submodules).
%   \ Inspector \rightarrow_{modifies} JSON\_Scene\_File\ 
% \end{math}

% % \begin{math}
% %   (\exists \ Terminal \in Editor.submodules).
% %   \ Terminal = Terminal
% % \end{math}



% The interactions between different modules can be represented using functions and mappings.
% Let \( f: E \to H \) represent the function mapping elements from the Editor to the Hub.
% Similarly, \( g: H \to S_p \) represents the mapping from Hub to Splash.


% \begin{math}

%   user \xrightarrow{launch} splash     
%        \xrightarrow{launch} Hub          
%        \xrightarrow{launch} Editor
%        \xrightarrow{launch} \{Hierarchy, SceneView, Inspector, Assets \}

%   SceneView \xrightarrow{launch} (\text{c++ } \land python ) \  Runner \xrightarrow{launch} \{ opengl, matplotlib, numpy \}

%   Hierarchy \xrightarrow{reads} \{ active_scene.json \}
  
%   Inspector \xrightarrow{reads} \{ active_game_object.json \}
  
%   Assets    \xrightarrow{reads} \{ FileTreeOf(projectPath) \}

%   (user \land Hub ) \lor (user \land Editor) \lor (user \land CLInterface) \xrightarrow{Request} (FileManager \lor SceneManager)

%   \xrightarrow{request} (FileManager \lor SceneManager)
%         \xrightarrow{response} 

%   %      \rightarrow_{request} FileManager \rightarrow_{response} Editor \rightarrow_{reaction} user

%   % user \rightarrow splash     \rightarrow_{action} Hub  \rightarrow_{request} FileManager \rightarrow_{response} Hub \rightarrow_[reaction] user

%   % user \rightarrow_{request} Hub.CLI    \rightarrow_{request} FileManager 

%   % user \rightarrow_{request} Editor.CLI \rightarrow_{request} FileManager 

%   % user \rightarrow_{request} FileManager 
% \end{math}






        \section*{Implementation}
          





    \begin{center}
    \includegraphics[width=1 \textwidth]{app_arch.png}
    \end{center}








    \part{Bibliography}
      \chapter*{Bibliography}

        \section*{Literature}
          In the following, i will present all the literature i used in this research and it will be grouped based on what i learned from that said resource.

          \subsection*{Mathematical Framework}
            \subsubsection*{"Mathematics For Game Developers" by AUTHOR}
              \input{src/content/study/book/GameDevMath}
          \subsection*{Rendering Engine}
            \subsubsection*{"Computer Graphics in C/C++" by AUTHOR}
              \input{src/content/study/book/CompGraph}
            \subsubsection*{"opengl SUPERBIBLE" by AUTHOR}
              \input{src/content/study/book/superbible}
              \textbf{How does the screen know what the videoboard wants to render?}
                \input{src/content/study/youtube/beneater}
          \subsection*{C++ Software Infrastructure}
            \subsubsection*{C++ by BJORN STROUPWAFFLE}
              \input{src/content/study/book/bjorn}

        \section*{Field Study}
          \subsection*{CS189 Stanford Computer Graphics Student Projects}
            \input{src/content/study/paper/stanford/CS189}
          \subsection*{CS201 Stanford Machine Learning Student Projects}
            \input{src/content/study/paper/stanford/CS201}
          \subsection*{Simulating Human Emunacra}
            \input{src/content/study/paper/emulatra}
          \subsection*{Detective}
            This technology had been expanded further from research environments, this technology made it into production and there now exists services that allow access to fine-tuned ai models with special integration for popular use-cases.
            \input{src/content/study/provider/detective}
          \subsection*(How does the screen know what to display?)







    % % \blankpage
    % 




\part{Overview}
% \addcontentsline{toc}{part}{Overview}
  % I am now on the edge and await.
  "University teaches you more than what to think. It teaches you how to think" Stefan Pantiru
  
  \chapter{Introduction} 
% \addcontentsline{toc}{chapter}{Introduction}

    \section{Motivation}
        I chose this thesis project because of my extended knowledge around the subject and because i thought i could combine all of my past research into this.

        Before i even started writing, i already had experience working in the following fields and i will briefly present my computer science background

        \subsection{Background}
            The following sections highlight pieces of my work that are relevant to this thesis. Each included project is significant due to specific implementations that are directly or indirectly connected to this paper.

            \subsubsection{Computer Graphics Experience}
                \begin{SCfigure}[0.9][h] 
                    \caption{Fractal Tree Visualization
                    
                    (usage of p5.js primitive functions in a recursive manner)

                    \href{https://www.youtube.com/watch?v=ajd0GGZgnDg&list=PL-j3UE1st04BZqRXq6eUBHpovhKjA1kiX&index=7}{showcase}}
                    
                    \includegraphics[width=8cm]{fractal.png}
                    \centering
                \end{SCfigure}                    
                \begin{itemize} 
                    \item Supershape Rendering Techniques 
                    \item Function Visualizer in OpenGL 
                    \item Complex Function Visualizer 
                \end{itemize}

            \subsubsection{Game Development Experience} 
                \begin{itemize} 
                    \item 3D Open World Environment Development 
                    \item Studies of Vector Movements in Unity 
                \end{itemize}

                % \subsubsection{in computer graphics}
            %     \subsubsection{fractal tree}
            %     \subsubsection{supershapes}
            %     \subsubsection{visualising basic function in opengl}
            %     \subsubsection{moving towards complex functions (mandelbrot series, julia fatou series)}
            %     \subsubsection{3D rendering}

            \subsubsection{in game development}
                my game development studies had mostly been around 3D open-world games. Being fascinated by rockstar's grand theft auto series i want to build something similar. i always wondered how cj was able to move in all the directions and calculating how there would be way too many paths to generate all the possible outcomes. so there should be smarter ways to do movement.
                And there is. Using vectors.
                My unity projects were mostly about perfectioning the 3D vector movement. Something that i also tried to implement in the opengl framework.   

        \subsection{Goal}
            I challenged myself to dig deeper into Game Development. I wanted to understand what makes all the pretty images move. i already had somewhat of an understanding of how the frames have to be processed independently and displayed in a fastly manner in order to trick the brain. but i wanted to go deeper then that.

            i already understood how to do certain simple tasks in unity, but i was so fascinated of the "transform.position = Vector2.One * scalar" command that i wanted to create a similar environment. 




            the intention of this project is to act as a foundation for a possible game engine that i will continue to deveop in the future. 

            a game engine is no easy task, there are many running parts and each of them must be SOLID. 

        % \subsection{Progress}

        %     until now i have the opengl framework in a stable state, this journey gave me a deeper understanding of all the running parts needed. 

        %     Now that i've been able to replicate some of the foundational concepts in enclosed environments, in the future i tend to use some already existing frameworks for physics and scene management and also ml libraries like scipy.

    \section{Literature Review}
            
            in order to achieve this project i have went through multiple pieces of literature.

            the ones i used most extensively are:
                \subsection{Books}
                    \subsubsection{OpenGL SuperBible}
                        SuperBible provided a thorough introduction to OpenGL,
                        detailing its functions and capabilities. This resource
                        was instrumental in understanding the core principles of
                        rendering and shading, which are fundamental to the
                        development of any graphics application. By following
                        the examples and exercises in this book, I was able to
                        implement efficient rendering pipelines and gain a deep
                        understanding of shader programming.
                    \subsubsection{Computer Graphics (Donald Hearn)}
                        % this gave me an in-depth understanding of the computer graphics field and also very insightful insides to primitives, drawing algorithms, popular solutions to popular problems. 
                        Donald Hearn's Computer Graphics offered a comprehensive overview of graphics
                        primitives and the algorithms used to draw them. The book's clear explanations
                        of line drawing algorithms, polygon filling techniques, and transformations were
                        particularly beneficial. Implementing these algorithms in my application allowed
                        me to create accurate and efficient rendering routines.
                    \subsubsection{Mathematics for Game Development (Christopher Tremblay)}
                        % i held this book very closely while writing the math engine for the game-engine.
                        % i recaped my veVctorial knowledge and understood which operators are most relevant in such projects and why. 
                        Christopher Tremblay's book on
                        mathematics for game development provided a solid
                        foundation in vectorial math, which is crucial for tasks
                        such as collision detection and physics simulation. The
                        detailed explanations of vector operations, matrix
                        transformations, and geometric algorithms were directly
                        applied in the development of the vector math library in
                        my application.
                    \subsubsection{C++ (Bjarne Stroustrup)}
                        Bjarne Stroustrup's definitive
                        guide to C++ significantly improved my programming
                        skills, enabling me to write efficient, robust, and
                        maintainable code. The book's coverage of advanced C++
                        features, such as templates, polymorphism, and the
                        Standard Template Library (STL), was particularly
                        valuable in structuring my application and optimizing:want
                        performance.
                \subsection{Papers}
                    \subsubsection{ml agents that can communicate to one another}
                    \subsubsection{comp graphics projects from stanford}
                    \subsubsection{...}

    \section{Theorethical overview}
        \subsection{Electronics}
        \subsection{Math}
        \subsection{Computer Graphics}
        \subsection{Game Development}
        \subsection{Coding Practices}


% The first ever game created was 'Tennis for Two' and was played on an oscilloscope. From then, gaming evolved from simple pixelated experiences to complex, immersive digital worlds.

% "Before game engines, games were typically written as singular entities: a game for the Atari 2600, for example, had to be designed from the bottom up to make optimal use of the display hardware (...) 
% Even on more accommodating platforms, very little could be reused between games."

% (Game Engine, Wikipedia)

% % (Did u know that the first Roller Coaster Tycoon was written completly in Assembly?)

% Programmers needed a way to make the game building process more efficient. So around the mid-1990s, thanks to Epic Games and their launch of the Unreal Engine and thanks to Id Software's Doom and Quake games the term "Game Engine" started to become more and more popular.

% Fast forward mid-2020s, now we have access to ultra realistic tank simulators for soldier training ( projectName ), surreal worlds filled with fantasy creatures ( Middle Earth: Shadow Of Mordor ) and even indie projects like Hyerbolica that portraits how a non-euclidian world would behave like. Projects like these would've been way harder ( if not actually impossible ) to pull off without the help of game engines.

% \pagebreak

% \section{Game Engines}
% "The line between a game and its engine is often blurry. Some engines make a reasonably clear distinction, while others make almost no attempt to separate the two. ( ... ) We should probably reserve the term
% 'game engine' for software that is extensible and can be used as the foundation for many different games without major modification."

% ( Game Engine Architecture. by Jason Gregory )

% \begin{figure}[!h]
%     \centering
%     \includegraphics[width=1\linewidth]{chapters/gameEngine Usability.png}
%     \caption{Game Engine Reusability Gamut}
%     \label{fig:Game-Engine Reusability}
% \end{figure}



% Behind the curtains of any interactive application there is most likely a Game Engine. 
% You have probably heard before about Unity and Unreal but there are many more game engines out there. Some of them are In-House, some of them are Open-Source, some of them are made for a specific Genre. But none of them is offering Built-In Machine Learning Integration. 

% \subsection{What does a Game Engine offer?}

% While not limited to, some of the tools we expect a game engine to offer are: 
% \begin{enumerate}
%     \item Input Handling
%     \item Player Mechanics
%     \item Game-Specific Rendering
%     \item Collision \& Physics
%     \item Audio Playback
%     \item Game Cameras
%     \item Ai \& Behaviour Management
%     \item Online Multiplayer
%     \item Scene Graph
%     \item Scripting System
%     \item Visual Effects
% \end{enumerate}

% \section{What is Machine Learning?}

% \begin{enumerate}
%     \item predicts text, predicts what the next word in a sentence would most likely be.
%     \item it's too broad of a topic to be able to explain in detail in just one paper. And it's also not the purpose of this paper. This paper only wants to use the power of ai to simulate better npc's in games.
%     \item for the purpose of this, i will not try to recreate or train an machine learning model, i will use an already existing one: GPT 3.5-Turbo by OpenAI.
% \end{enumerate}

% \subsection{Machine Learning in Game Engines}
% \begin{enumerate}
%     \item Inworld Origins are already working on a solution for implementing GPT4 to be used in a NPC dialogue scenario.
%     \item There is this study where multiple GPT4 agents that can simulate believable human behavior. (Generative Agents: Interactive Simulacra of Human Behavior)
%     % https://arxiv.org/pdf/2304.03442.pdf
%     \item Currently, there are no game engines that have Machine Learning Systems built-in. 
% \end{enumerate}

% \section{Scope and Objectives of the Thesis}
% \begin{enumerate}
%     \item exploration, development, and evaluation of a game engine that integrates GPT-based machine learning capabilities 
%     \item explore the architecture of a game engine while building a basic one in Python using OpenGL.
%     \item explore what it means to add machine learning capabilities to a game engine. 
%     \item The focus of this research is on leveraging the power of natural language processing (NLP) and GPT models to enhance various aspects of game development, including storytelling, character interactions, procedural content generation, and player experiences.
% \end{enumerate}



  
\chapter*{Motivation}
\addcontentsline{toc}{chapter}{Motivation}

  In the previous i mentioned the toolbox that this university allowed me to organise for myself. In the following i will go over some of the tools i specifically tailered for this implementation.

  The biggest motivator factor for persuing this project is a strong personal interest in the field of game development and computer graphics.

  My goal became to apply some of those concepts aquired during my study in my personal research.

  The way i chose to apply these concepts is by building a graphics framework compatible with some of the simpler machine learning solutions.


  
\chapter*{Goals}
\addcontentsline{toc}{chapter}{Goal}

  The goal is to build an Abstraction Layer that opens the inner workings of computer graphics and machine learning.

  This Abstraction Layer \textbf{must} respect PHIGS standards, \textbf{must} be easy-to-learn for both intermediate and beginner end-users, \textbf{must} present shorter and more straight-forward solution than existing solutions.

  The project's theme is combining Machine Learning with the Game Makers' Environments and standards.

  In the following i will present the benefits of this theme.



  
\chapter*{Benefits of Game Engines}
\addcontentsline{toc}{chapter}{Game Engines}
  Building your own game engine is the standard practice when it comes to big in-house teams. 
  This of course, comes with both advantages and disadvantages.
  The advantages are spaced around the ideas of security and integrity.
  While the disadvantages are mostly cost-related.

  The often solution for the smaller companies remains open-source software.
  There are a few providers on the market already, if names like Unity, Unreal, godot might sound familiar, then maybe also do names like P5.js, Processing or maybe even Coding Train, Sebastian Lague, Code Bullet all the way towards electronics field with Ben Eater. 

  It doesn't matter where in the previous enumeration you lost the references because the following will go more in-depth on the relevant discoveries of each of them.


  \section*{Benefits of building your own game engine}
      \subsection*{Standard industry practice}
          it is common for big companies to build their own in-house game engines and then develop their games on it.
          advantages:     provides competitive edge, security, integrity ...
          disadvantages:  cost, team special for that.

          it is common for smaller sized companies to develop their games/projects on already existing game engines  
          advantages:     already existing reources and docs, community, 
          disadvantages:  dificult to come up with unique style.

          On the following, i want us to analyse some of the game engines that there are. and draw out relevant particularities of each of them.

          for this i have chosen 1 Open Source graphics framework (p5.js), 1 closed software game engine (RAGE) and 1 restricted game engine (UNITY).

          \subsubsection*{RAGE}
              even though this is a closed project and unaccesible to the public, over the years different screenshots and code snippets had been leaked and/or reverse-engineered and i would like us to take a look at some of the more expressive ones.
          \subsubsection*{Unity}
              Even though unity's source code is not accesible to the public, the engine is completly free to use for any individual*.
          \subsubsection*{P5.js}
              This graphics engine is completly free and \href[]{github.com}{open-source}

      \subsection*{Educational purposes}
          i strongly believe that building a game engine had massively improved my abilities.

  \section*{Benefits of integrating machine learning with gaming}

      ml is the new and fancy cool shiny thing that shows promising numbers and gets ppl hyped and everyone loves it and it must be implemented into everything that exists.

      game development is no exception.

      \subsection*{Simulating Human Interaction}
          NPCs are important in games.

          NPCs are there to guide the player and are the projection the game designers into the game world. 

          Because of this, is is really important that npcs have fluid dialogue and dont break the illusion of choice too easily.

          Current solutions imply using dialogue trees.
          
          they can still feel rough on the edges. and the illusion can be broken easily when u have to decide from a set of predefined dialogue choices.

          the imersiveness of games could greatly improve if ml were to be implemented on top of this already existing dialogue tree solution.

          Such solutions have already been experimented with, in the following i will present the findings of 3 other papers that use machine learning to improve npc dialogue and interaction.
          Two of the following are solutions for human-to-ai dialogue and one of them simulates ai-to-ai.
          \subsubsection*{Ai interacting to Ai}
              % TODO: Refactor and fact-check with paper.
              One paper that i found extremily fascinating was \href[]{google.com}{TITLE} by AUTHOR. They created an environment that allowed ml agents to communicate to one another. One of the most exciting outcomes was that one agent organised a birthday party and proceeded to invite other ml agents to the party. In the following i will briefly go over the implementatation design for one agent:
              
              % there were cool images of tables for the datastructures used. they had a timetable and there was a complex way for managing memories and long-time storage of relevant info. 

          \subsection*{Ai interacting to humans}
              Another paper that highlights machine-learning agents interacting in human-like behaviour is \href[]{google.com}{TITLE} by AUTHOR. This team even offers multiple solutions for implementing such agents in popular environments such as Unity or ??.

              One popular demo of their plugin? is the game \href[]{google.com}{GAMENAME}. 
              Game that illustrates a scenario where the player is a detective and has to figure out a case, with the added twist that comunicating with any of the non-playable-characters (NPCs) is made through the microphone and with openai dialogue. 
              
              There is also a mod for the popular game Skyrim that allows the player to have fluid dialogue with any in-game character. 
      
      \subsection*{Out-performing Human Performance}
          popular youtuber Code Bullet has a series where he "solves" games using AI models. He usually uses neural-networks for his solutions. One recent such video is where he programmed a JUMP KING ml.
          
          There are chess bots being developed that use machine learning in an attempt to "solve" the game of chess. So it is clear to say there is is a lot of incentivise towards acomodating machine learning algorithms into games.







    % 



\part{Field Study}
% \addcontentsline{toc}{part}{Field Study}


  



\chapter*{Field Study - Introduction}
\addcontentsline{toc}{chapter}{Introduction}
  The Field Study chapters will present a brief explication of the moving parts that are involved in computer graphics rendering.

  First chapter is the one with more focus on embedded-systems concepts 
  Second chapter goes through graphics abstraction layers 
  Third chapter presents an overview of the popular solutions offered to end-users. 
  \section*{Real-World use cases}
  % \addcontentsline{toc}{section}{use-cases}
      \subsection*{Student Projects - Stanford Computer Graphics Course}
          LINK
      \subsection*{Academic Research}
          IMAGES


  


\chapter*{Field Study - Hardware}
\addcontentsline{toc}{chapter}{Hardware}
  This chapter brings focus towards embedded systems by being a brief description of the process of getting the LEDs on our screens to display whatever the computer's video board decides to render. 
  \section*{The connection between screen and motherboard}
  % \addcontentsline{toc}{section}{Video cards}
    \subsection*{Proof of concept using Arduino}
      TINKERCAD IMAGE
        % \href{https://www.youtube.com/watch?v=l7rce6IQDWs}{video}
    \subsection*{Proof of concept using embedded circuits}
      BEN EATER VIDEO
    \subsection*{videoboards}
      LINUS TECH TIPS OLD VIDEO MANUFATEURS PCBs


  

\chapter*{Field Study - Software}
\addcontentsline{toc}{chapter}{Software}
  This chapter accepts the embedded solutions as they are and develops solutions that step forward. 
  The usual philosophy is developing abstraction layers.

    \section*{OPENGL - Motivation}
    % \addcontentsline{toc}{section}{opengl}
        There are a few options when choosing a graphics abstraction solution. The most popular in the game development industry are Vulkan and DirectX. DirectX is more appropriate when it comes to Windows-specific optimizations, while Vulkan profits from Low-level control and performance and performs better at High-performance applications with multi-threading.

        The only disadvantage to both those solution is that neither of them is as documented as opengl. Also, opengl is more popular in the educational/academic space and felt like the more appropriate choice.

    \section*{OPENGL - Features}
        OpenGL's extensive documentation comes with both positives and negatives. Being one of the oldest solution to this problem it had seen multiple refactoring stages throught the years, this is best observed when realising there is a new revision of the opengl superbible released every couple of years. Each presenting the usual "How-to" projects and also acting as an update journal.

        \pagebreak

        In the following i will briefly talk about popular opengl features.


  
        \subsection*{PHIGS Standard Compliant}
            Offers primitive support and for their attributes by using Rendering Algorithms that use the low-level per pixel drawing functions extensively.

            \subsubsection*{Primitives}
                \begin{itemize}
                  \item Points and Lines
                  \begin{itemize}
                    \item Width
                    \item Color
                    \begin{itemize}
                      \item Flood-Fill Algorithms
                    \end{itemize}
                    \item Segments
                  \end{itemize}
                  \item Circle-generating Algorithms
                  \item Text Renderer
                \end{itemize}

            \subsubsection*{Operations}
              \begin{itemize}
                \item Matrices
                \begin{itemize}
                  \item Homogeneous Representation
                  \item Matrix      Multiplication
                \end{itemize}

                \item Basic Transformations
                \begin{itemize}
                    \item Translation
                    % TODO: ADD EQUASIONS
                    \item Rotation
                    \begin{itemize}
                        \item Euler Method
                        \item Euler Problem
                        \item Quaternions
                        % TODO: EQUASIONS
                    \end{itemize}
                    \item Scaling
                    % TODO: EQUASIONS
                \end{itemize}
                \end{itemize}


    \section*{Other Graphics Abstraction Layers}
    % \addcontentsline{toc}{section}{Other Graphics Abstraction Layers}
        \subsection*{Vulkan}
        % \addcontentsline{toc}{subsection}{Vulkan}
        \subsection*{directX}
        % \addcontentsline{toc}{subsection}{DirectX}










    % \part{Implementation}
% \addcontentsline{toc}{part}{Implementation}




  
\chapter*{Mathematical Framework}

This mathematical framework is very important because it assures that the phigs requirements can be fulfiled and how to.

\section*{Homogenous Representation}

Intuitively, for storing a 3D point one might think about using a vector of length 3 ( maybe call them x, y, z ) and have a good day.
Well, this datastructure is good \emph{enough} for most cases.
There is, of course, one little edge case in one of the things we must be able to do with this DataStructure that would benefit of storing a 3D point as a 4D Vector (x, y, z, \textbf{\emph{w}}).

\textbf{Please do not be afraid of the \emph{w}.  Since \emph{w} is defined by a pretty straight-forward formula.} 

w=1, if point and 
w=0, if arrow

% TODO: MATH FORMULA FOR w = 1, if point w = 0, if arrow

\subsection*{Vector Multiplication}
\subsection*{Matrix Multiplication}

\section*{Coordinate Systems}

There are two main coordinate systems:
\begin{itemize}
  \item Cartesian Coordinates
  \item Polar     Coordinates 
\end{itemize}

Formulas for converting from one to another:
\begin{itemize}
  \item from Polar     to Cartesian
  \begin{itemize}
    \item cos(x)
    \item sin(x)
  \end{itemize}
  \item from Cartesian to Polar
  \begin{itemize}
    \item x
    \item y
  \end{itemize}
\end{itemize}


In the following, we will take a look over the possible operation that are extensively used:

In most of the cases, the equasions are really easy.

\subsection*{Translation}
\subsection*{Scalation}
\subsection*{Rotation}
\subsubsection*{Euler-Lock}
And it would've been all so easy if it weren't for you! 

Euler Lock is a problem that ocurs when we try to use euler coordinates in rotation aplications. This problem can become extremily dangerous when solving robotics solutions where you can't afford to ???

\subsubsection*{Quaternions}
For eliminating the euler-lock problem, quaternions are used. Quaternions are ??

Formulas:








  


\chapter*{Implementation - Architecture}
\addcontentsline{toc}{chapter}{Architecture}

  In case you haven't read the previous chapter i advise glimsing over the chapters' titles at least once because it would give better context on where my project lives in the graphics lifecycle.

  My Application is a Graphics Abstraction Layer that imitates industry standards when it comes to procedures used for ease of learning.
  Besides being an easy-to-use beginner-friendly tool because of following industry standard solution in the c++ rendering backend engine. At it's core, this rendering engine is built on top of a vectorial mathematics engine.

  The novelty that this project brings to the computer graphics world is the presence of a python Machine Learning backend that acts as an abstraction layer for simplifying the communication with services (openai, ollama) or with powerful machine learning libraries (scipy, skilearn) 
  
  \begin{figure}
    \begin{center}
      \includegraphics[width=\textwidth]{implementation_arch.png}
    \caption{Arch}
    \end{center}
  \end{figure}
  \pagebreak


  \section*{The backend}
    The Backend is wrote mostly in C++ and offers the user namespaces to interact with the opengl inner-workings of the engine.

    \begin{figure}
      \begin{center}
        \includegraphics[width=\textwidth]{implementation_arch.png}
      \caption{Arch}
      \end{center}
    \end{figure}
    \pagebreak

  \section*{The Frontend}

    The frontend is wrote mostly in Python with the help of PyQt, a python wrapper for the industry standard qt.

    \begin{figure}
      \begin{center}
        \includegraphics[width=\textwidth]{implementation_arch.png}
      \caption{Arch}
      \end{center}
    \end{figure}
    \pagebreak

  \section*{Additional Tools}

    Additional tools has been developed already, tools that improves the programmer's workflow.
    Such example is a web scraper for data collecting.
    Docs for each major component.
    \begin{figure}
      \begin{center}
        \includegraphics[width=\textwidth]{implementation_arch.png}
      \caption{Arch}
      \end{center}
    \end{figure}
    \pagebreak

  \chapter*{Implementation - Showcase}
\addcontentsline{toc}{chapter}{Showcase}













    % 
\part{Results}





  


\chapter*{Technical Review}
\addcontentsline{toc}{chapter}{Technical Review}


    In this chapter, we will inspect to what extent needs one piece of software to satisfy in order for it to be considered part of the collection containing game engine.
    
    \section*{Speed}
    % \addcontentsline{toc}{section}{Speed}
    \section*{Memory}
    % \addcontentsline{toc}{section}{Memory}


    \paragraph*{Wikipedia} "A game engine is a software framework primarily designed for the development of video games and generally includes relevant libraries
    % and support programs such as a level editor.
    (...).
    % Game engine can also refer to the development software supporting this framework, typically 
    % a suite of tools and features for developing games.
    The core functionality typically provided by a game engine may include a rendering engine ("renderer") for 2D or 3D graphics, a physics engine or collision detection (and collision response), sound, scripting, artificial intelligence, 
    (...)
    % networking, streaming, memory management, threading, localization support, scene graph, and video support for cinematics.
    "

    So, in order to satisfy this definition, a piece of software \emph{P} can be considered a game engine, if and only if \emph{P} satisfies the following:

    \begin{itemize}
        \item \emph{P} is a software framework
        \begin{itemize}
          \item {"A software framework is an abstraction in which software, providing generic functionality, can be selectively changed by additional user-written code. 
            (...) 
            It provides a standard way to build and deploy applications and is a universal, reusable software environment 
            (...)
            to facilitate the development of software applications, products and solutions. "} source: Wikipedia

            % \paragraph*{Wikipedia} {"A software framework is an abstraction in which software, providing generic functionality, can be selectively changed by additional user-written code.
            % % , thus providing application-specific software. 
            % (...)
            % It provides a standard way to build and deploy applications and is a universal, reusable software environment 
            % % that provides particular functionality as part of a larger software platform 
            % (...)
            % to facilitate the development of software applications, products and solutions. "}
            \begin{itemize}
                \item Generic functionality that can be selectively adapted based on user's code.
                \item provides a standard way of building and deploying applications
            \end{itemize}
        \end{itemize}
        \item \emph{P} includes a suite of relevant engines
    \end{itemize}






  


\chapter*{Future Improvements}
\addcontentsline{toc}{chapter}{Future Improvements}
  This has been a journey and after reading this paper you should have a view through the window of progress. There are still many to implement and properly integrate. There will be multiple update journals of this type posted on the following.

\section*{Packaged builds}
  The scope of this project is to one day make it as an AUR package and also a PyPi library.

  Some work has already been made in this direction in the matter that one of the early builds is available by running 'pip install -i https://test.pypi.org/simple/ game-genie' in any terminal.
  Unfortunately, I had to interrupt pip support for until application grows into a more stable and mature form.

  % \input{src/chapters/results/}



    % \chapter{Bibliography} 
% \addcontentsline{toc}{chapter}{Bibliography}

\begin{itemize}
    \item Author1, \textit{Book1}, 2018
    \item Author2, \textit{Boook2}, 2017
\end{itemize}


    % % \chapter{Introduction} 
% \addcontentsline{toc}{chapter}{Introduction}

    \section{Motivation}
        I chose this thesis project because of my extended knowledge around the subject and because i thought i could combine all of my past research into this.

        Before i even started writing, i already had experience working in the following fields and i will briefly present my computer science background

        \subsection{Background}
            The following sections highlight pieces of my work that are relevant to this thesis. Each included project is significant due to specific implementations that are directly or indirectly connected to this paper.

            \subsubsection{Computer Graphics Experience}
                \begin{SCfigure}[0.9][h] 
                    \caption{Fractal Tree Visualization
                    
                    (usage of p5.js primitive functions in a recursive manner)

                    \href{https://www.youtube.com/watch?v=ajd0GGZgnDg&list=PL-j3UE1st04BZqRXq6eUBHpovhKjA1kiX&index=7}{showcase}}
                    
                    \includegraphics[width=8cm]{fractal.png}
                    \centering
                \end{SCfigure}                    
                \begin{itemize} 
                    \item Supershape Rendering Techniques 
                    \item Function Visualizer in OpenGL 
                    \item Complex Function Visualizer 
                \end{itemize}

            \subsubsection{Game Development Experience} 
                \begin{itemize} 
                    \item 3D Open World Environment Development 
                    \item Studies of Vector Movements in Unity 
                \end{itemize}

                % \subsubsection{in computer graphics}
            %     \subsubsection{fractal tree}
            %     \subsubsection{supershapes}
            %     \subsubsection{visualising basic function in opengl}
            %     \subsubsection{moving towards complex functions (mandelbrot series, julia fatou series)}
            %     \subsubsection{3D rendering}

            \subsubsection{in game development}
                my game development studies had mostly been around 3D open-world games. Being fascinated by rockstar's grand theft auto series i want to build something similar. i always wondered how cj was able to move in all the directions and calculating how there would be way too many paths to generate all the possible outcomes. so there should be smarter ways to do movement.
                And there is. Using vectors.
                My unity projects were mostly about perfectioning the 3D vector movement. Something that i also tried to implement in the opengl framework.   

        \subsection{Goal}
            I challenged myself to dig deeper into Game Development. I wanted to understand what makes all the pretty images move. i already had somewhat of an understanding of how the frames have to be processed independently and displayed in a fastly manner in order to trick the brain. but i wanted to go deeper then that.

            i already understood how to do certain simple tasks in unity, but i was so fascinated of the "transform.position = Vector2.One * scalar" command that i wanted to create a similar environment. 




            the intention of this project is to act as a foundation for a possible game engine that i will continue to deveop in the future. 

            a game engine is no easy task, there are many running parts and each of them must be SOLID. 

        % \subsection{Progress}

        %     until now i have the opengl framework in a stable state, this journey gave me a deeper understanding of all the running parts needed. 

        %     Now that i've been able to replicate some of the foundational concepts in enclosed environments, in the future i tend to use some already existing frameworks for physics and scene management and also ml libraries like scipy.

    \section{Literature Review}
            
            in order to achieve this project i have went through multiple pieces of literature.

            the ones i used most extensively are:
                \subsection{Books}
                    \subsubsection{OpenGL SuperBible}
                        SuperBible provided a thorough introduction to OpenGL,
                        detailing its functions and capabilities. This resource
                        was instrumental in understanding the core principles of
                        rendering and shading, which are fundamental to the
                        development of any graphics application. By following
                        the examples and exercises in this book, I was able to
                        implement efficient rendering pipelines and gain a deep
                        understanding of shader programming.
                    \subsubsection{Computer Graphics (Donald Hearn)}
                        % this gave me an in-depth understanding of the computer graphics field and also very insightful insides to primitives, drawing algorithms, popular solutions to popular problems. 
                        Donald Hearn's Computer Graphics offered a comprehensive overview of graphics
                        primitives and the algorithms used to draw them. The book's clear explanations
                        of line drawing algorithms, polygon filling techniques, and transformations were
                        particularly beneficial. Implementing these algorithms in my application allowed
                        me to create accurate and efficient rendering routines.
                    \subsubsection{Mathematics for Game Development (Christopher Tremblay)}
                        % i held this book very closely while writing the math engine for the game-engine.
                        % i recaped my veVctorial knowledge and understood which operators are most relevant in such projects and why. 
                        Christopher Tremblay's book on
                        mathematics for game development provided a solid
                        foundation in vectorial math, which is crucial for tasks
                        such as collision detection and physics simulation. The
                        detailed explanations of vector operations, matrix
                        transformations, and geometric algorithms were directly
                        applied in the development of the vector math library in
                        my application.
                    \subsubsection{C++ (Bjarne Stroustrup)}
                        Bjarne Stroustrup's definitive
                        guide to C++ significantly improved my programming
                        skills, enabling me to write efficient, robust, and
                        maintainable code. The book's coverage of advanced C++
                        features, such as templates, polymorphism, and the
                        Standard Template Library (STL), was particularly
                        valuable in structuring my application and optimizing:want
                        performance.
                \subsection{Papers}
                    \subsubsection{ml agents that can communicate to one another}
                    \subsubsection{comp graphics projects from stanford}
                    \subsubsection{...}

    \section{Theorethical overview}
        \subsection{Electronics}
        \subsection{Math}
        \subsection{Computer Graphics}
        \subsection{Game Development}
        \subsection{Coding Practices}


% The first ever game created was 'Tennis for Two' and was played on an oscilloscope. From then, gaming evolved from simple pixelated experiences to complex, immersive digital worlds.

% "Before game engines, games were typically written as singular entities: a game for the Atari 2600, for example, had to be designed from the bottom up to make optimal use of the display hardware (...) 
% Even on more accommodating platforms, very little could be reused between games."

% (Game Engine, Wikipedia)

% % (Did u know that the first Roller Coaster Tycoon was written completly in Assembly?)

% Programmers needed a way to make the game building process more efficient. So around the mid-1990s, thanks to Epic Games and their launch of the Unreal Engine and thanks to Id Software's Doom and Quake games the term "Game Engine" started to become more and more popular.

% Fast forward mid-2020s, now we have access to ultra realistic tank simulators for soldier training ( projectName ), surreal worlds filled with fantasy creatures ( Middle Earth: Shadow Of Mordor ) and even indie projects like Hyerbolica that portraits how a non-euclidian world would behave like. Projects like these would've been way harder ( if not actually impossible ) to pull off without the help of game engines.

% \pagebreak

% \section{Game Engines}
% "The line between a game and its engine is often blurry. Some engines make a reasonably clear distinction, while others make almost no attempt to separate the two. ( ... ) We should probably reserve the term
% 'game engine' for software that is extensible and can be used as the foundation for many different games without major modification."

% ( Game Engine Architecture. by Jason Gregory )

% \begin{figure}[!h]
%     \centering
%     \includegraphics[width=1\linewidth]{chapters/gameEngine Usability.png}
%     \caption{Game Engine Reusability Gamut}
%     \label{fig:Game-Engine Reusability}
% \end{figure}



% Behind the curtains of any interactive application there is most likely a Game Engine. 
% You have probably heard before about Unity and Unreal but there are many more game engines out there. Some of them are In-House, some of them are Open-Source, some of them are made for a specific Genre. But none of them is offering Built-In Machine Learning Integration. 

% \subsection{What does a Game Engine offer?}

% While not limited to, some of the tools we expect a game engine to offer are: 
% \begin{enumerate}
%     \item Input Handling
%     \item Player Mechanics
%     \item Game-Specific Rendering
%     \item Collision \& Physics
%     \item Audio Playback
%     \item Game Cameras
%     \item Ai \& Behaviour Management
%     \item Online Multiplayer
%     \item Scene Graph
%     \item Scripting System
%     \item Visual Effects
% \end{enumerate}

% \section{What is Machine Learning?}

% \begin{enumerate}
%     \item predicts text, predicts what the next word in a sentence would most likely be.
%     \item it's too broad of a topic to be able to explain in detail in just one paper. And it's also not the purpose of this paper. This paper only wants to use the power of ai to simulate better npc's in games.
%     \item for the purpose of this, i will not try to recreate or train an machine learning model, i will use an already existing one: GPT 3.5-Turbo by OpenAI.
% \end{enumerate}

% \subsection{Machine Learning in Game Engines}
% \begin{enumerate}
%     \item Inworld Origins are already working on a solution for implementing GPT4 to be used in a NPC dialogue scenario.
%     \item There is this study where multiple GPT4 agents that can simulate believable human behavior. (Generative Agents: Interactive Simulacra of Human Behavior)
%     % https://arxiv.org/pdf/2304.03442.pdf
%     \item Currently, there are no game engines that have Machine Learning Systems built-in. 
% \end{enumerate}

% \section{Scope and Objectives of the Thesis}
% \begin{enumerate}
%     \item exploration, development, and evaluation of a game engine that integrates GPT-based machine learning capabilities 
%     \item explore the architecture of a game engine while building a basic one in Python using OpenGL.
%     \item explore what it means to add machine learning capabilities to a game engine. 
%     \item The focus of this research is on leveraging the power of natural language processing (NLP) and GPT models to enhance various aspects of game development, including storytelling, character interactions, procedural content generation, and player experiences.
% \end{enumerate}


    % % 
\chapter*{Motivation}
\addcontentsline{toc}{chapter}{Motivation}

  In the previous i mentioned the toolbox that this university allowed me to organise for myself. In the following i will go over some of the tools i specifically tailered for this implementation.

  The biggest motivator factor for persuing this project is a strong personal interest in the field of game development and computer graphics.

  My goal became to apply some of those concepts aquired during my study in my personal research.

  The way i chose to apply these concepts is by building a graphics framework compatible with some of the simpler machine learning solutions.


    % 
    % % \chapter{Computer Graphics}

    \section{Math Engine}
        \subsection{vector.cpp}
            the Vector3 class supports operations such as addition, subtraction, dot product, and cross product, enabling users to perform complex calculations with ease.
        \subsection{random.cpp}
            Generating true-randomness is one of the biggest computer science challenges. 

            In my project i needed a way the user to be able to get a "insert formal specs here for non-deterministic random" vector3 and color variable. 

            Lukily, i have stumbled upon this random-generator function and was able to implement it. Now, in the framework there is a function that returns a different each call random variable and also the seed is randomized so it differes between individual launches of the application. 
    \section{Renderer Engine}
        \subsection{primitives}
            OpenGL and WebGL are quite similar. P5.js is built on top of WebGL and offers the end-user (besides many more) a set of convenient functions for simple tasks like drawing the background, drawing a square, a circle, choosing stroke width and color.
        
            besides these functions i have thought of implementing classes for each of the base geometrical shapes. This implementation will come in handy when integrating with the game engine stuff (see chapter3:Game Development, section GameObjects). For example, each renderer component will use one of the primitive classes (square, circle, box, sphere) to display itself on the screen.
            
        \subsection{Transformations}
            As we've already mentioned in the mathematics chapter, transformations are a big deal when dealing with game development. They basically give fluidity. Also, they can be a tough challenge to overcome, especially because it requires the mathematical framework to be able to do complex calculations and as quickly as possible. 

            This topic is brought up multiple times in this paper and in this chapter we will focus on the specifics on how primitives transform.
            
            types of transformation:
            \begin{itemize}
                \item translation
                \item rotation
                \item scaling
                \item skewing
                \item warping
                \item ...
            \end{itemize}

            fun fact: did u know that u can achieve a rotation by doing multiple skew transformations in a row?
        \subsection{color.cpp}
    \section{Collision Detection}



% \section{Titlul secțiunii 1}

% \begin{figure}
%     \centering
%     \includegraphics[width=0.5\linewidth]{chapters/pyroGamerHub.png}
%     \caption{Enter Caption}
%     \label{fig:enter-label}
% \end{figure}

% Id donec ultrices tincidunt arcu non sodales neque. Integer eget aliquet nibh praesent. Euismod in pellentesque massa placerat duis ultricies lacus sed. Mauris ultrices eros in cursus turpis massa. Integer quis auctor elit sed vulputate mi. Nibh ipsum consequat nisl vel pretium lectus quam id leo. Vel elit scelerisque mauris pellentesque pulvinar pellentesque. Suscipit tellus mauris a diam maecenas. Ultrices eros in cursus turpis massa tincidunt. Tristique senectus et netus et malesuada fames ac turpis egestas. Suspendisse interdum consectetur libero id faucibus nisl tincidunt eget. Sed risus pretium quam vulputate dignissim suspendisse in. Donec adipiscing tristique risus nec feugiat in fermentum posuere. A lacus vestibulum sed arcu non odio euismod lacinia at.

% \section{Titlul secțiunii 2}

% Pellentesque pulvinar pellentesque habitant morbi tristique senectus et. Ornare suspendisse sed nisi lacus sed viverra tellus in hac. Non sodales neque sodales ut etiam sit. In hendrerit gravida rutrum quisque non. Diam quam nulla porttitor massa id neque aliquam. Diam sit amet nisl suscipit adipiscing bibendum est ultricies integer. Cras fermentum odio eu feugiat pretium nibh ipsum. Egestas integer eget aliquet nibh praesent tristique magna. Porttitor eget dolor morbi non arcu risus quis varius quam. Gravida rutrum quisque non tellus orci. Diam volutpat commodo sed egestas egestas.
    % % \chapter{Game Development}
    \section{Game Engine}
        \subsection{GameObjects}
        \subsection{Components}
    \section{Collision Detection}
    % % \chapter{Machine Learning}
    \section{Probabilities}
    \section{Scipy compatibility}
    % 
    % % \input{src/chapters/conclusions}

\end{document}
