


\chapter*{Technical Review}
\addcontentsline{toc}{chapter}{Technical Review}


    In this chapter, we will inspect to what extent needs one piece of software to satisfy in order for it to be considered part of the collection containing game engine.
    
    \section*{Speed}
    % \addcontentsline{toc}{section}{Speed}
    \section*{Memory}
    % \addcontentsline{toc}{section}{Memory}


    \paragraph*{Wikipedia} "A game engine is a software framework primarily designed for the development of video games and generally includes relevant libraries
    % and support programs such as a level editor.
    (...).
    % Game engine can also refer to the development software supporting this framework, typically 
    % a suite of tools and features for developing games.
    The core functionality typically provided by a game engine may include a rendering engine ("renderer") for 2D or 3D graphics, a physics engine or collision detection (and collision response), sound, scripting, artificial intelligence, 
    (...)
    % networking, streaming, memory management, threading, localization support, scene graph, and video support for cinematics.
    "

    So, in order to satisfy this definition, a piece of software \emph{P} can be considered a game engine, if and only if \emph{P} satisfies the following:

    \begin{itemize}
        \item \emph{P} is a software framework
        \begin{itemize}
          \item {"A software framework is an abstraction in which software, providing generic functionality, can be selectively changed by additional user-written code. 
            (...) 
            It provides a standard way to build and deploy applications and is a universal, reusable software environment 
            (...)
            to facilitate the development of software applications, products and solutions. "} source: Wikipedia

            % \paragraph*{Wikipedia} {"A software framework is an abstraction in which software, providing generic functionality, can be selectively changed by additional user-written code.
            % % , thus providing application-specific software. 
            % (...)
            % It provides a standard way to build and deploy applications and is a universal, reusable software environment 
            % % that provides particular functionality as part of a larger software platform 
            % (...)
            % to facilitate the development of software applications, products and solutions. "}
            \begin{itemize}
                \item Generic functionality that can be selectively adapted based on user's code.
                \item provides a standard way of building and deploying applications
            \end{itemize}
        \end{itemize}
        \item \emph{P} includes a suite of relevant engines
    \end{itemize}





