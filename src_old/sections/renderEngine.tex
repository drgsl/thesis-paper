
\section*{Color32}


\begin{lstlisting}
class color32
{
  public:
    unsigned int r, g, b, a;

    color32(double grayscale) 
           : color(grayscale, grayscale, grayscale, 1.0f) { }
    color32(double _r, double _g, double _b, double _a = 1.0f) {  }
};
\end{lstlisting}




As you can see, similar to the vector class, the color datastructure is also a 4D vector. This will come in handy when dealing with shaders. 


  \subsection*{Opposite Colors}
  \begin{lstlisting}
  class color32
  {
    public:
      unsigned int r, g, b, a;

      color32(double grayscale) 
             : color(grayscale, grayscale, grayscale, 1.0f) { }
      color32(double _r, double _g, double _b, double _a = 1.0f) {  }
  };



  \end{lstlisting}


  \begin{lstlisting}
  void start();
  void update();
  int main() { Awake(); }
  void Awake() {
    RenderEngine::setStart(start);
    RenderEngine::setUpdate(update);
    RenderEngine::setFixedUpdate(fixedUpdate);
    // MUST BE CALLED LAST
    RenderEngine::START(true);
  }
  void start() {
    Gameobject go;
    go.transform.position = Random::Vector3().normalised *  Random::Value(-5, 5); 
    Debug::Log(go.transform.position)
  }
  \end{lstlisting}


\section*{Primitives}

% PHIGS (Programmer's Hierarchical Interactive Graphics System) provides the function:

\begin{lstlisting}

  point(x, y, c); // Changes the color of the pixel at location <x, y> to c

  line(<<x>, <y>>, 
       <<x>, <y>>); // Draws a line between 2 points

  background(color);

  square(point1, point2);
  fill(color);
  circle(point , radius);

  noStroke();
\end{lstlisting}

% \begin{equation}
  % P(x, y), \forall x, y, -\frac{\text{screen::WIDTH}}{2} \leq x \leq \frac{\text{screen::WIDTH}}{2}
  % \label{eq:pixel}
% \end{equation}

These functions are the fundamental building blocks for rendering graphics on a screen in OpenGL. 
For the existance of these functions, 
OpenGL implements various procedures to draw shapes and perform graphical operations.

\subsection*{Points and Lines}

Points and lines are basic graphical primitives. A point is represented by its coordinates \((x, y)\) on the screen, and a line is represented by a linear equation.

\subsubsection*{Line-Drawing Procedure}

A line can be represented by the equation:

\begin{equation}
  y = mx + b
\end{equation}

where \( m \) is the slope of the line, calculated as:

\begin{equation}
  m = \frac{\Delta y}{\Delta x}
\end{equation}

% % ### DDA Algorithm (Digital Differential Analyzer)

% The DDA algorithm is an incremental method for drawing lines:

% \[
% \begin{align*}
%   \Delta x &= x_1 - x_0 \\
%   \Delta y &= y_1 - y_0 \\
%   \text{If } |\Delta x| > |\Delta y|, &\text{ then } \text{steps} = |\Delta x| \text{ else } \text{steps} = |\Delta y| \\
%   \Delta x_{\text{increment}} &= \frac{\Delta x}{\text{steps}} \\
%   \Delta y_{\text{increment}} &= \frac{\Delta y}{\text{steps}}
% \end{align*}
% \]

% Starting from \((x_0, y_0)\), the algorithm increments the coordinates:

% \[
% \begin{align*}
%   x &= x + \Delta x_{\text{increment}} \\
%   y &= y + \Delta y_{\text{increment}}
% \end{align*}
% \]

% % ### Bresenham's Line Drawing Algorithm

% Bresenham's algorithm uses integer arithmetic for efficient computation:

% \[
% \begin{align*}
%   \Delta x &= x_1 - x_0 \\
%   \Delta y &= y_1 - y_0 \\
%   \text{Decision parameter} \, p &= 2\Delta y - \Delta x
% \end{align*}
% \]

% For each \(x\) from \(x_0\) to \(x_1\):

% \[
% \begin{align*}
%   \text{If } p < 0, &\text{ then } p = p + 2\Delta y \\
%   \text{else}, &\text{ increment } y \text{ and } p = p + 2(\Delta y - \Delta x)
% \end{align*}
% \]

% % ### Line Attributes

% % #### Width

% Line width can be controlled using a parameter \(w\):

% \begin{equation}
%   \text{LineWidth}(w)
% \end{equation}

% % #### Color

% Color can be specified using RGB values:

% \begin{equation}
%   \text{Color}(r, g, b)
% \end{equation}

% % #### Segments

% A line segment between two points \((x_0, y_0)\) and \((x_1, y_1)\) can be drawn using the above algorithms, with the line attributes applied.

% \subsection*{Circle-generating Algorithms}

% % ### Midpoint Circle Algorithm

% The midpoint circle algorithm efficiently draws circles using the symmetry of circles and incremental calculations:

% \[
% \begin{align*}
%   x &= r, y = 0 \\
%   \text{Decision parameter} \, p &= 1 - r
% \end{align*}
% \]

% For each \(x\) from \(r\) to 0:

% \[
% \begin{align*}
%   \text{If } p < 0, &\text{ then } p = p + 2x + 3 \\
%   \text{else}, &\text{ decrement } y \text{ and } p = p + 2(x - y) + 5
% \end{align*}
% \]

% % ### Parametric Form

% Using the parametric form of a circle:

% \[
% \begin{align*}
%   x &= r \cos \theta \\
%   y &= r \sin \theta
% \end{align*}
% \]

% \subsection*{Flood-Fill Algorithms}

% Flood-fill algorithms fill a contiguous area of pixels with a specified color.

% % ### Boundary Fill Algorithm

% Starting from a seed point \((x, y)\):

% \[
% \begin{align*}
%   \text{If } P(x, y) \neq \text{boundary\_color} \text{ and } P(x, y) \neq \text{fill\_color}, &\text{ then fill } P(x, y) \text{ and apply to neighboring pixels}
% \end{align*}
% \]

% \subsection*{Text Renderer}

% Rendering text involves drawing each character glyph using the above primitives and positioning them appropriately.

% \section*{Conclusion}

% The described primitives and algorithms form the basis of graphical rendering in OpenGL, allowing the creation of complex images from basic elements like points, lines, and circles.













%       % \begin{itemize}
%       %   \item Points and Lines
%       %   \begin{itemize}
%       %     \item Width
%       %     \item Color
%       %     \begin{itemize}
%       %       \item Flood-Fill Algorithms
%       %     \end{itemize}
%       %     \item Segments
%       %   \end{itemize}
%       %   \item Circle-generating Algorithms
%       %   \item Text Renderer
%       % \end{itemize}

%   % \subsubsection*{Operations}
%   %   \begin{itemize}
%   %     \item Matrices
%   %     \begin{itemize}
%   %       \item Homogeneous Representation
%   %       \item Matrix      Multiplication
%   %     \end{itemize}

%   %     \item Basic Transformations
%   %     \begin{itemize}
%   %         \item Translation
%   %         % TODO: ADD EQUASIONS
%   %         \item Rotation
%   %         \begin{itemize}
%   %             \item Euler Method
%   %             \item Euler Problem
%   %             \item Quaternions
%   %             % TODO: EQUASIONS
%   %         \end{itemize}
%   %         \item Scaling
%   %         % TODO: EQUASIONS
%   %     \end{itemize}
%   %     \end{itemize}







% \section*{Game Engine Component}

% \subsection*{Render Engine}

% \begin{verbatim}
% namespace RenderEngine
% {
%     void Enabled(bool state);
 
%     void setStart (void (*func)());
 
%     void setUpdate(void (*func)());
% }
% \end{verbatim}

% The Render Engine namespace includes functions for enabling or disabling the engine, and setting start and update functions. These functions are represented as follows:

% \[
% \text{Enabled}(state) \rightarrow \text{void}
% \]

% \[
% \text{setStart}(\text{func}) \rightarrow \text{void}
% \]

% \[
% \text{setUpdate}(\text{func}) \rightarrow \text{void}
% \]

% Here, \(\text{func}\) is a function pointer that can be assigned to start or update functions for the engine.

% \subsection*{Vector3 Class}

% \begin{verbatim}
% class Vector3
% {
% public:
%   double x, y, z, w;
%   static Vector3 zero;
%   static Vector3 one;

%   Vector3() : Vector3(0, 0, 0, 1) {}
 
%   Vector3(double _x, double _y = 0.0f, double _z = 0.0f, double _w = 1.0f) 
%     : x(_x), y(_y), z(_z), w(_w) {}
 
%   Vector3& operator=(const Vector3& other);
%   Vector3& operator+=(const Vector3& other);
%   Vector3& operator*=(double scalar);
%   Vector3 operator*(double scalar) const;
%   Vector3 operator-(const Vector3& other) const; 
%   Vector3 operator+(const Vector3& other) const; 
%   double dot(const Vector3& other) const;
%   Vector3 operator/(double scalar) const;
% };
% \end{verbatim}

% Vector operations for the Vector3 class can be represented mathematically as follows:

% \[
% \mathbf{v} = \begin{pmatrix} x \\ y \\ z \\ w \end{pmatrix}
% \]

% Addition:
% \[
% \mathbf{v}_1 + \mathbf{v}_2 = \begin{pmatrix} x_1 \\ y_1 \\ z_1 \\ w_1 \end{pmatrix} + \begin{pmatrix} x_2 \\ y_2 \\ z_2 \\ w_2 \end{pmatrix} = \begin{pmatrix} x_1 + x_2 \\ y_1 + y_2 \\ z_1 + z_2 \\ w_1 + w_2 \end{pmatrix}
% \]

% Dot product:
% \[
% \mathbf{v}_1 \cdot \mathbf{v}_2 = x_1 x_2 + y_1 y_2 + z_1 z_2 + w_1 w_2
% \]

% Scalar multiplication:
% \[
% \mathbf{v} \cdot s = \begin{pmatrix} x \\ y \\ z \\ w \end{pmatrix} \cdot s = \begin{pmatrix} sx \\ sy \\ sz \\ sw \end{pmatrix}
% \]

% \subsection*{Color Class}

% \begin{verbatim}
% class Color
% {
% public:
%   float r, g, b, a;
%   /* 4bit color support */
%   static Color black;
%   static Color white;
%   static Color red;
%   static Color green;
%   static Color blue;
%   static Color yellow;
%   static Color cyan;
%   static Color magenta;
%   static Color gray;
%   static Color darkGray;
%   static Color lightGray;
%   static Color darkRed;
%   static Color darkGreen;
%   static Color darkBlue;
%   static Color brown;
%   static Color orange;
 
%   Color(double _value)
%     : Color(_value, _value, _value, 1.0f) {}
 
%   Color(double _r, double _g, double _b)
%     : Color(_r, _g, _b, 1.0f) {}
 
%   Color(double _r, double _g, double _b, double _a)
%     : r(_r), g(_g), b(_b), a(_a) {}
% };
% \end{verbatim}

% Colors in the Color class can be represented mathematically as:

% \[
% \mathbf{c} = \begin{pmatrix} r \\ g \\ b \\ a \end{pmatrix}
% \]

% Where \(r\), \(g\), \(b\), and \(a\) represent the red, green, blue, and alpha (transparency) components of the color, respectively.

% \subsection*{Component Namespace}

% \begin{verbatim}
% namespace Component
% {
%   class Transform
%   {
%     public:
%       Vector3 position; 
%       Vector3 rotation; 
%       Vector3 scale;    
%   };

%   class Rigidbody
%   {
%     public:
%       Vector3 velocity;
%   };

%   class Renderer
%   {
%     public:
%       Color color;
%   };
% }
% \end{verbatim}

% The Component namespace includes classes for Transform, Rigidbody, and Renderer components. Mathematically, these components can be represented as follows:

% \[
% \text{Transform} = \begin{pmatrix} \text{position} \\ \text{rotation} \\ \text{scale} \end{pmatrix}
% \]

% \[
% \text{Rigidbody} = \text{velocity}
% \]

% \[
% \text{Renderer} = \text{color}
% \]

% Where position, rotation, and scale are Vector3 objects, velocity is a Vector3 object, and color is a Color object.

% \subsection*{GameObject Class}

% \begin{verbatim}
% class GameObject
% {
% public:
%   Component::Transform transform;
%   Component::Rigidbody rigidbody;
%   Component::Renderer renderer;

%   GameObject();
% };
% \end{verbatim}

% A GameObject is composed of a Transform, Rigidbody, and Renderer component. This can be represented as:

% \[
% \text{GameObject} = \begin{pmatrix} \text{Transform} \\ \text{Rigidbody} \\ \text{Renderer} \end{pmatrix}
% \]

% Where each component is as defined in the Component namespace, representing the position, rotation, scale, velocity, and color of the GameObject.

